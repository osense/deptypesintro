\documentclass[12pt]{article}
\usepackage[mathletters]{ucs}
\usepackage[utf8x]{inputenc}
\usepackage{proof}
\usepackage{amsmath}
\usepackage{latexsym}


\title{Correspondece between Second-order logic and~Polymorphic lambda calculus}
\author{Adam Krupicka\\
        Faculty of Informatics\\
        Masaryk University, Brno
}
\date{23.5.2016}


\begin{document}
\maketitle

\begin{abstract}
ABSTRACT
\end{abstract}

\section{Introduction}
Into…

\section{Simply typed lambda calculus}
In this section we will remind the reader of the well-known correspondence between Natural deduction and the Simply typed λ calculus. We won't go into much detail here, as this is covered well in other literature.
\subsection{Natural deduction}
We will consider only the fragment of Natural deduction consisting of conjunction and implication. The proof system for this fragment is defined as follows:
\paragraph{Introduction rules}
$$
\infer[\mathrm{∧I}]{A ∧ B}{A & B}
\qquad
\infer[\mathrm{⇒I}]{A → B}{\infer*{B}{[A]}}
$$
\paragraph{Elimination rules}
$$
\infer[\mathrm{∧₁E}]{A}{A ∧ B}
\qquad
\infer[\mathrm{∧₂E}]{B}{A ∧ B}
\qquad
\infer[\mathrm{⇒E}]{B}{A & A → B}
$$
\subsection{Simply typed λ calculus}
The definition of the calculus's terms is inductive:
\begin{itemize}
    \item A variable $x:T$ is a term of type T,
    \item If $u:U$ and $v:V$ are terms, then $⟨u,v⟩:(U × V)$ is a term (pair), 
    \item If $t:(U × V)$ is a term, then $(π₁ \ t):U$ is a term (first projection),
    \item If $t:(U × V)$ is a term, then $(π₂ \ t):V$ is a term (second projection),
    \item If $t:T$ is a term, then $λx:U.t$ is a term of type U → V (abstraction),
    \item If $u:(V → T)$ and $v:V$ are terms, then $(u \ v):T$ is a term (application).
\end{itemize}
\paragraph{Evaluation}
Evaluation can be understood as simple rewriting of terms, according to these equations:
\[(λx:U.t) \ u → t[x:=u] \qquad π₁ \ ⟨u,v⟩ → u \qquad π₂ \ ⟨u,v⟩ → v\]
\subsection{The correspondence}
This can be observed between proof trees of natural deduction and types of terms constructed inside the calculus.
\begin{itemize}
    \item $\mathrm{∧I}$ corresponds to the creation of pairs,
    \item $\mathrm{∧₁E}$ to first projection,
    \item $\mathrm{∧₂E}$ to second projection,
    \item $\mathrm{⇒I}$ to abstraction,
    \item $\mathrm{⇒E}$ to application.
\end{itemize}
\subsection{Pairs in the Simply typed λ calculus}
\label{crappairs}
It should be noted that, from the calculus's point of view, it isn't necessary to explicitly define pairs as we have done above. Pairs can be easily defined from within the calculus:
\begin{align*}
    \mathrm{pair} &≡ λa:ℕλb:ℕλx:(ℕ → ℕ → ℕ).x \ a \ b\\
    \mathrm{fst} \ t &≡ t \ (λx:ℕλy:ℕ.x)\\
    \mathrm{snd} \ t &≡ t \ (λx:ℕλy:ℕ.y)
\end{align*}
However, such a construct would only work for pairs of a single type (here the atomic type of natural numbers, ℕ). In addition, the correspondence between the rules associated with ∧ and pairs is clearer in this way.

\section{System F}
In the previous section, we have described the basic correspondence between proofs and types. To this effect, we have utilized only a few logical operators. One might naturally be curious about what happens when we attempt to introduce back some of the unconsidered operators. We will start with the ∀ quantifier, as this is perhaps the easiest one.
\subsection{∀ fragment of Second-order logic}
Introducing the ∀ quantifier would bring us into a fragment of First-order logic. However, we will go further than this — we will allow quantifying not only over variables, but also over sets. The rationale for this is that on the computational side, quantifying over variables makes little sense. What would it mean to say that $λx:ℕ.x$ is the identity function for all $x$ (of type ℕ)?
\paragraph{Introduction and elimination rules}
$$
\infer[\mathrm{∀I}]{∀X.A}{A}
\qquad
\infer[\mathrm{∀E}]{A[X:=t]}{∀X.A}
$$
In the case of the $\mathrm{∀I}$ rule, it is necessary to ensure that the variable $X$ is not free in $A$.
\subsection{Back to the calculus}
On the computational side, this gives us abstraction over types, written Λ. We can therefore extend the definition of Simply typed λ calculus thus:
\begin{itemize}
    \item If $t:T$ is a term and X is not free in T, then $Λt:(∀X.T)$ is a term (universal abstraction),
    \item If $u:(∀X.U)$ is a term and $V$ is a type, then $(u \ V):U[X:=V]$ is a term (universal application).
\end{itemize}
\paragraph{Evaluation}
While we will soon see that we can drop the evaluation rules for pairs, however we need a new rule for universal abstraction and application:
\[(∀X.T) \ U → T[X:=U]\]
\paragraph{Example}
We can use this to generalize our example with pairs from~\ref{crappairs}:
\begin{align*}
    \mathrm{pair} &≡ λx:Uλy:VΛZλz:(U → V → Z).x \ a \ b\\
    \mathrm{fst} \ t &≡ t \ U \ (λx:Uλy:V.x)\\
    \mathrm{snd} \ t &≡ t \ V \ (λx:Uλy:V.y)
\end{align*}
Let us evaluate an example term:
\begin{align*}
    \mathrm{fst} \ (\mathrm{pair} \ p₁ \ p₂) =& \ \mathrm{fst} \ (λa:Aλb:BΛXλx:(A → B → X).x \ a \ b) \ p₁ \ p₂\\
    \leadsto& \ \mathrm{fst} \ (ΛXλx:(A → B → X).x \ p₁ \ p₂)\\
    =& \ (ΛXλx:(A → B → X).x \ p₁ \ p₂) \ A \ (λx:Aλy:B.x)\\
    \leadsto& \ (λx:(A → B → A).x \ p₁ \ p₂) \ (λx:Aλy:B.x)\\
    \leadsto& \ (λx:Aλy:B.x) \ p₁ \ p₂\\
    \leadsto& \ p₁\\
\end{align*}
\end{document}

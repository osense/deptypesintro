\documentclass[12pt]{article}
\usepackage[mathletters]{ucs}
\usepackage[utf8x]{inputenc}
\usepackage[english]{babel}
\usepackage{proof}
\usepackage{amsmath}
\usepackage{amsthm}
\usepackage{latexsym}

\newtheorem{theorem}{Theorem}
\newtheorem{definition}{Definition}
\newcommand{\true}{\enskip\mathrm{true}}
\newcommand{\term}{\enskip\mathrm{term}}

\title{Dependent types}
\author{Adam Krupicka\\
        Faculty of Informatics\\
        Masaryk University, Brno
}
\date{23.5.2016}


\begin{document}
\maketitle

\begin{abstract}
ABSTRACT
\end{abstract}


\section{Introduction}
…


\section{Definitions}

To say that some variable is \textbf{free} in a type means that it does not occur anywhere in it (e.g.\ $A$ is free in $B$ or $B → B$, but not in $A → B → B$).

\textbf{$\mathcal{L} (\mathcal{M})$} denotes the language of some calculus \mathcal{M}, that is, all the terms that can be constructed from some fixed starting set of variables.

\section{First-order predicate logic}


\subsection{Natural deduction}
We will consider a fragment consisting only of implication and the quantifiers. Why this is sufficient will become clear in the next section.
\paragraph{Hypothesis}
\[A\]
\paragraph{Implication}
$$
\infer[\mathrm{⇒I^u}]{A ⇒ B \true}{\infer*{B \true}{\infer[u]{A \true}}}
\qquad
\infer[\mathrm{⇒E}]{B \true}{A \true & A ⇒ B \true}
$$
The introduction rule on the left states that assuming some initial hypothesis $A$, if we then proceed to show $B$, then $A ⇒ B$ holds, however we must remember to discharge our initial hypothesis of $A$. It can be thought of as taking the entire derivation tree of $B$ from the hypothetical $A$ and stating it more succinctly as $A ⇒ B$.

The elimination rule on the right should be obvious.
\paragraph{Universal quantification}
$$
\infer[\mathrm{∀I^u}]{∀x.A \true}{\infer*{A[x:=t] \true}{\infer[u]{t \term}}}
\qquad
\infer[\mathrm{∀E}]{A[x:=t] \true}{∀x.A \true & t \term}
$$
The introduction rule is easier to read bottom to top: if we wish to conclude $∀x.A$, then we are required to show that if we replace every instance of $x$ in $A$ with some arbitrary term $t$, $A$ still holds. The term $t$ can be thought of as a variable, because we are not allowed to assume anything about it — it is just some $t$.

The rule on the right is, again, rather obvious — it allows us to specialize a universally qualified formula back into some term.
\paragraph{Existential quantification}
$$
\infer[\mathrm{∃I^u}]{∃x.A[t:=x] \true}{A \true}
\qquad
\infer[\mathrm{∃E^{u,v}}]{B \true}{∃x.A \true & \infer*{B \true}{\infer[v]{A[x:=t] \true}{\infer*{}{\infer[u]{t \term}}}}}
$$
Here, the introduction rule simply states that if we have a proposition $A$ containing some term $t$ which happens to be true, then we can abstract away from the specific $t$ and simply conclude the existence of some $x$ where $A$ (undergoing the replacement $t:=x$) is true.

The elimination rule says that if, having a proof of $∃x.A$, we can construct a proof of $A$ (modulo replacement) with some arbitrary term $t$ (which is really just a variable), and furthermore we then proceed to show that $B$ follows from $A$, we can conclude $B$.


\section{Dependent types}


\section{The Curry-Howard correspondence}


\section{Dependent types in practice}


\end{document}
